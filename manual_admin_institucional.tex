\documentclass{article}

\usepackage[utf8]{inputenc}
\usepackage[spanish]{babel}
\usepackage{graphicx}
\usepackage{hyperref}
\usepackage{geometry}
\usepackage{enumitem}
\usepackage{xcolor}
\usepackage{titlesec}
\usepackage{fancyhdr}
\usepackage{float}
\usepackage{caption}

\geometry{a4paper, margin=2.5cm}

% Colores SENA
\definecolor{senaverde}{RGB}{0, 175, 0}
\definecolor{senaverdeoscuro}{RGB}{0, 136, 0}

% Configuración de títulos
\titleformat{\section}
{\Large\bfseries\color{senaverde}}
{}
{0em}
{}[\titlerule]

\titleformat{\subsection}
{\large\bfseries}
{}
{0em}
{}

% Encabezado y pie de página
\pagestyle{fancy}
\fancyhf{}
\fancyhead[L]{\textcolor{senaverde}{\textbf{SGDIS}}}
\fancyhead[R]{\textcolor{gray}{Manual de Usuario - Admin Institucional}}
\fancyfoot[C]{\thepage}
\renewcommand{\headrulewidth}{0.4pt}

\title{SGDIS - Manual de Usuario\\Administrador Institucional}
\author{Julian Chaparro Barrera}
\date{Diciembre 2025}

\begin{document}

\maketitle

\newpage
\tableofcontents
\newpage

\section{Introducción}

Este manual de usuario está diseñado específicamente para los usuarios con rol de \textbf{Administrador Institucional (ADMIN\_INSTITUTION)} del Sistema de Gestión de Inventario SENA (SGDIS). El manual proporciona instrucciones detalladas sobre cómo utilizar todas las funcionalidades disponibles para este rol.

\subsection{Alcance del Rol de Administrador Institucional}

El rol de Administrador Institucional tiene las siguientes características:

\begin{itemize}
    \item \textbf{Alcance geográfico:} Puede gestionar únicamente los recursos, usuarios e inventarios de su institución asignada.
    \item \textbf{Permisos:} Tiene acceso completo a todas las funcionalidades del sistema dentro de su institución.
    \item \textbf{Restricciones:} No puede gestionar usuarios con rol SUPERADMIN ni ADMIN\_REGIONAL, ni puede acceder a datos de otras instituciones o regionales.
    \item \textbf{Gestión de usuarios:} Puede crear y gestionar usuarios con roles: Warehouse y User (dentro de su institución).
    \item \textbf{Gestión de inventarios:} Puede gestionar todos los inventarios de su institución.
    \item \textbf{Sin acceso a Centros:} No puede visualizar ni gestionar regionales o instituciones.
\end{itemize}

\section{Acceso al Dashboard de Administrador Institucional}

Para acceder al dashboard de administrador institucional:

\begin{enumerate}
    \item Inicie sesión en el sistema con credenciales de administrador institucional.
    \item Después del inicio de sesión, será redirigido automáticamente al dashboard de administrador institucional.
    \item El dashboard se encuentra en la ruta \texttt{/admin\_institution/dashboard}.
\end{enumerate}

\begin{figure}[H]
    \centering
    \includegraphics[width=0.9\textwidth]{SuperAdminImage/dashboardSgdis1.png}
    \caption{Dashboard principal del Administrador Institucional}
    \label{fig:dashboard-admin-institution}
\end{figure}

\subsection{Panel de Navegación}

El panel lateral izquierdo contiene las siguientes opciones de navegación:

\begin{itemize}
    \item \textbf{Dashboard:} Vista general de la institución con estadísticas y gráficos.
    \item \textbf{Inventario:} Gestión de inventarios de la institución.
    \item \textbf{Usuarios:} Administración de usuarios de la institución.
    \item \textbf{Transferencias:} Administración de transferencias entre inventarios de la institución.
    \item \textbf{Verificación:} Gestión de verificaciones de items de la institución.
    \item \textbf{Préstamos:} Administración de préstamos de items de la institución.
    \item \textbf{Bajas:} Gestión de cancelaciones de items de la institución.
    \item \textbf{Reportes:} Generación y visualización de reportes de la institución.
    \item \textbf{Auditoría:} Registro de actividades del sistema en la institución.
    \item \textbf{Notificaciones:} Gestión de notificaciones del sistema.
    \item \textbf{Importar/Exportar:} Funciones de importación y exportación de datos de la institución.
    \item \textbf{Configuración:} Configuraciones generales del sistema.
    \item \textbf{Mi Perfil:} Información personal y configuración de cuenta.
\end{itemize}

\begin{figure}[H]
    \centering
    \includegraphics[width=0.2\textwidth]{SuperAdminImage/sideBarSgdis.png}
    \caption{Panel de navegación lateral del Administrador Institucional}
    \label{fig:panel-navegacion-inst}
\end{figure}

\subsection{Dashboard de Administrador Institucional}

El dashboard proporciona una vista general de la institución con las siguientes secciones:

\subsubsection{Tarjetas de Estadísticas}

El dashboard muestra tarjetas con estadísticas clave de la institución:

\begin{itemize}
    \item \textbf{Total Inventarios:} Muestra el número total de inventarios en la institución, con desglose de activos e inactivos.
    \item \textbf{Total Usuarios:} Cantidad total de usuarios registrados en la institución, con distribución por roles (Admin Inst., Warehouse, Usuarios).
    \item \textbf{Total Transferencias:} Cantidad de transferencias con estado (pendientes, aprobadas, rechazadas) dentro de la institución.
    \item \textbf{Total Verificaciones:} Número total de verificaciones realizadas en la institución.
    \item \textbf{Total Préstamos:} Cantidad de préstamos registrados en la institución.
    \item \textbf{Total Items:} Número total de items en todos los inventarios de la institución, con valor total en COP.
    \item \textbf{Notificaciones No Leídas:} Cantidad de notificaciones pendientes por leer.
\end{itemize}

\begin{figure}[H]
    \centering
    \includegraphics[width=0.95\textwidth]{SuperAdminImage/estadisticasSgdis.png}
    \caption{Tarjetas de estadísticas del dashboard}
    \label{fig:dashboard-estadisticas-inst}
\end{figure}

\subsubsection{Gráficos y Visualizaciones}

El dashboard incluye los siguientes gráficos específicos de la institución:

\begin{enumerate}
    \item \textbf{Usuarios por Rol:} Gráfico circular (doughnut) que muestra la distribución de usuarios según su rol (Admin Inst., Warehouse, Usuarios) en la institución.
    
    \item \textbf{Bajas por Inventario:} Gráfico de barras que muestra la cantidad de cancelaciones (bajas) por inventario en la institución.
    
    \item \textbf{Transferencias por Estado:} Gráfico circular que muestra la distribución de transferencias según su estado (Pendientes, Aprobadas, Rechazadas) en la institución.
    
    \item \textbf{Items por Inventario:} Gráfico de barras que muestra la cantidad de items por cada inventario de la institución.
\end{enumerate}

\begin{figure}[H]
    \centering
    \includegraphics[width=0.95\textwidth]{SuperAdminImage/estadisticasSgdis2.png}
    \caption{Gráficos y visualizaciones del dashboard}
    \label{fig:dashboard-graficos-inst}
\end{figure}

\subsubsection{Actividad Reciente}

El dashboard muestra tres secciones de actividad reciente de la institución:

\begin{itemize}
    \item \textbf{Inventarios Recientes:} Lista de los últimos 5 inventarios creados o modificados en la institución, mostrando nombre, institución y estado.
    \item \textbf{Transferencias Recientes:} Lista de las últimas 5 transferencias de la institución, mostrando el item, fecha y estado.
    \item \textbf{Verificaciones Recientes:} Lista de las últimas 5 verificaciones realizadas en la institución, mostrando el item y fecha.
\end{itemize}

\begin{figure}[H]
    \centering
    \includegraphics[width=0.95\textwidth]{SuperAdminImage/RECIENTES_SGDIS.png}
    \caption{Secciones de actividad reciente del dashboard}
    \label{fig:dashboard-actividad-inst}
\end{figure}

\newpage

\section{Gestión de Usuarios}

Como administrador institucional, puede gestionar todos los usuarios de su institución asignada.

\subsection{Acceso a Gestión de Usuarios}

\begin{enumerate}
    \item En el panel lateral, haga clic en \textbf{"Usuarios"}.
    \item Será redirigido a la página de gestión de usuarios (\texttt{/admin\_institution/users}).
\end{enumerate}

\begin{figure}[H]
    \centering
    \includegraphics[width=0.95\textwidth]{SuperAdminImage/usuarios_admin_dashboard.png}
    \caption{Página de gestión de usuarios}
    \label{fig:gestion-usuarios-inst}
\end{figure}

\subsection{Funcionalidades Disponibles}

\begin{itemize}
    \item \textbf{Ver lista de usuarios:} Visualice todos los usuarios registrados en su institución.
    \item \textbf{Buscar usuarios:} Utilice la barra de búsqueda para encontrar usuarios específicos.
    \item \textbf{Filtrar usuarios:} Filtre usuarios por rol y estado.
    \item \textbf{Crear nuevo usuario:} Agregue nuevos usuarios al sistema con roles Warehouse o User.
    \item \textbf{Editar usuario:} Modifique la información de usuarios existentes de la institución.
    \item \textbf{Cambiar rol de usuario:} Asigne o modifique el rol de un usuario (solo Warehouse o User).
    \item \textbf{Activar/Desactivar usuario:} Habilite o deshabilite cuentas de usuario de la institución.
    \item \textbf{Eliminar usuario:} Elimine usuarios del sistema (con precaución).
\end{itemize}

\subsection{Roles Disponibles para Crear}

\textbf{Importante:} Como Administrador Institucional, solo puede crear usuarios con los siguientes roles:

\begin{itemize}
    \item \textbf{Warehouse:} Usuarios encargados de la gestión de almacén y verificaciones.
    \item \textbf{User:} Usuarios estándar con permisos básicos de consulta y operación.
\end{itemize}

\textbf{No puede crear usuarios con roles:} ADMIN\_INSTITUTION, ADMIN\_REGIONAL ni SUPERADMIN.

\subsection{Crear Nuevo Usuario}

Para crear un nuevo usuario en su institución:

\begin{enumerate}
    \item Localice el botón \textbf{"Nuevo Usuario"} o \textbf{"Crear Usuario"} en la página de gestión de usuarios.
\end{enumerate}

\begin{figure}[H]
    \centering
    \includegraphics[width=0.6\textwidth]{SuperAdminImage/boton_crear_usuario_admin.png}
    \caption{Botón para crear nuevo usuario}
    \label{fig:boton-crear-usuario-inst}
\end{figure}

\begin{enumerate}
    \setcounter{enumi}{1}
    \item Haga clic en el botón para abrir el modal de creación de usuario.
    \item Complete el formulario del modal con la siguiente información:
    \begin{itemize}
        \item \textbf{Nombre completo:} Ingrese el nombre completo del usuario.
        \item \textbf{Correo electrónico:} Debe ser @soy.sena.edu.co o @sena.edu.co.
        \item \textbf{Cargo laboral:} Especifique el cargo del usuario.
        \item \textbf{Departamento laboral:} Indique el departamento al que pertenece.
        \item \textbf{Rol:} Seleccione el rol (Warehouse o User). \textbf{Nota:} Solo puede crear usuarios con estos roles.
        \item \textbf{Institución:} Se asigna automáticamente su institución.
        \item \textbf{Contraseña:} Establezca una contraseña inicial para el usuario.
        \item \textbf{Estado:} Seleccione si el usuario estará Activo o Inactivo.
    \end{itemize}
    \item Revise la información ingresada.
    \item Haga clic en \textbf{"Guardar"} o \textbf{"Crear"} para confirmar.
\end{enumerate}

\begin{figure}[H]
    \centering
    \includegraphics[width=0.85\textwidth]{SuperAdminImage/modal_crear_usuario_admin.png}
    \caption{Modal de creación de nuevo usuario con todos los campos del formulario}
    \label{fig:modal-crear-usuario-inst}
\end{figure}

\subsection{Editar Usuario}

Para editar un usuario existente de su institución:

\begin{enumerate}
    \item En la lista de usuarios, localice el usuario que desea editar.
    \item Localice el botón \textbf{"editar"} o el ícono de edición en la fila del usuario.
\end{enumerate}

\begin{figure}[H]
    \centering
    \includegraphics[width=0.7\textwidth]{SuperAdminImage/boton_editar_usuario_admin.png}
    \caption{Botón de editar usuario en la lista}
    \label{fig:boton-editar-usuario-inst}
\end{figure}

\begin{enumerate}
    \setcounter{enumi}{2}
    \item Haga clic en el botón para abrir el modal de edición.
    \item El modal mostrará el formulario con los datos actuales del usuario.
    \item Modifique los campos necesarios:
    \begin{itemize}
        \item Puede cambiar el nombre completo, cargo, departamento.
        \item Puede modificar el rol del usuario (solo Warehouse o User).
        \item Puede cambiar el estado (Activo/Inactivo).
    \end{itemize}
    \item Revise los cambios realizados.
    \item Haga clic en \textbf{"Guardar"} para aplicar los cambios.
\end{enumerate}

\begin{figure}[H]
    \centering
    \includegraphics[width=0.85\textwidth]{SuperAdminImage/modal_editar_usuario_admin.png}
    \caption{Modal de edición de usuario con los campos prellenados}
    \label{fig:modal-editar-usuario-inst}
\end{figure}

\subsection{Ver Detalles de Usuario}

Para visualizar la información completa de un usuario:

\begin{enumerate}
    \item En la lista de usuarios, localice el usuario del cual desea ver los detalles.
    \item Localice el botón \textbf{"Ver"} o el ícono de ojo en la fila del usuario.
\end{enumerate}

\begin{figure}[H]
    \centering
    \includegraphics[width=0.2\textwidth]{SuperAdminImage/ver_item_admin_icon.png}
    \caption{Botón para ver detalles de usuario}
    \label{fig:boton-ver-usuario-inst}
\end{figure}

\begin{enumerate}
    \setcounter{enumi}{2}
    \item Haga clic en el botón para abrir el modal de visualización.
    \item El modal mostrará toda la información del usuario:
    \begin{itemize}
        \item \textbf{Información personal:} Nombre completo, correo electrónico, foto de perfil.
        \item \textbf{Información laboral:} Cargo, departamento, institución.
        \item \textbf{Información del sistema:} Rol asignado, estado de la cuenta, fecha de registro.
        \item \textbf{Historial de actividad:} Últimas acciones realizadas en el sistema.
        \item \textbf{Inventarios asignados:} Lista de inventarios donde el usuario tiene acceso.
    \end{itemize}
\end{enumerate}

\begin{figure}[H]
    \centering
    \includegraphics[width=0.6\textwidth]{SuperAdminImage/visualizacion_usuario_admin.png}
    \caption{Modal de visualización de usuario con toda la información}
    \label{fig:modal-ver-usuario-inst}
\end{figure}

\subsection{Cambiar Contraseña de Usuario}

Para cambiar la contraseña de un usuario:

\begin{enumerate}
    \item En la lista de usuarios, localice el usuario al cual desea cambiar la contraseña.
    \item Localice el botón \textbf{"Cambiar Contraseña"} o el ícono de llave en la fila del usuario.
\end{enumerate}

\begin{figure}[H]
    \centering
    \includegraphics[width=0.2\textwidth]{SuperAdminImage/item_cambiar_contraseña_usuario.png}
    \caption{Botón para cambiar contraseña de usuario}
    \label{fig:boton-cambiar-contrasena-inst}
\end{figure}

\begin{enumerate}
    \setcounter{enumi}{2}
    \item Haga clic en el botón para abrir el modal de cambio de contraseña.
    \item El modal solicitará la siguiente información:
    \begin{itemize}
        \item \textbf{Nueva contraseña:} Ingrese la nueva contraseña para el usuario.
        \item \textbf{Confirmar contraseña:} Confirme la nueva contraseña ingresándola nuevamente.
    \end{itemize}
    \item Asegúrese de que ambas contraseñas coincidan.
    \item Haga clic en \textbf{"Guardar"} o \textbf{"Cambiar Contraseña"} para confirmar.
\end{enumerate}

\begin{figure}[H]
    \centering
    \includegraphics[width=0.5\textwidth]{SuperAdminImage/modal_cambiar_contraseña.png}
    \caption{Modal de cambio de contraseña de usuario}
    \label{fig:modal-cambiar-contrasena-inst}
\end{figure}

\subsection{Eliminar Usuario}

Para eliminar un usuario del sistema:

\begin{enumerate}
    \item En la lista de usuarios, localice el usuario que desea eliminar.
    \item Localice el botón \textbf{"Eliminar"} o el ícono de papelera en la fila del usuario.
\end{enumerate}

\begin{figure}[H]
    \centering
    \includegraphics[width=0.1\textwidth]{SuperAdminImage/boton_eliminar.png}
    \caption{Botón para eliminar usuario}
    \label{fig:boton-eliminar-usuario-inst}
\end{figure}

\begin{enumerate}
    \setcounter{enumi}{2}
    \item Haga clic en el botón para abrir el modal de confirmación de eliminación.
    \item El modal mostrará una advertencia indicando:
    \begin{itemize}
        \item Nombre del usuario a eliminar.
        \item Advertencia de que esta acción no se puede deshacer.
        \item Información sobre los datos que se eliminarán junto con el usuario.
    \end{itemize}
    \item Revise cuidadosamente la información antes de confirmar.
    \item Haga clic en \textbf{"Eliminar"} o \textbf{"Confirmar"} para eliminar el usuario permanentemente.
\end{enumerate}

\begin{figure}[H]
    \centering
    \includegraphics[width=0.5\textwidth]{SuperAdminImage/modal_eliminar_usuario_admin.png}
    \caption{Modal de confirmación de eliminación de usuario}
    \label{fig:modal-eliminar-usuario-inst}
\end{figure}

\subsection{Notas Importantes sobre Gestión de Usuarios}

\begin{itemize}
    \item Solo puede gestionar usuarios que pertenezcan a su institución asignada.
    \item No puede crear, editar ni asignar los roles ADMIN\_INSTITUTION, ADMIN\_REGIONAL o SUPERADMIN.
    \item Todos los usuarios que cree pertenecerán automáticamente a su institución.
    \item Solo puede ver usuarios de su propia institución.
\end{itemize}

\section{Gestión de Inventario}

El administrador institucional puede gestionar todos los inventarios de su institución.

\subsection{Acceso a Gestión de Inventario}

\begin{enumerate}
    \item En el panel lateral, haga clic en \textbf{"Inventario"}.
    \item Será redirigido a la página de gestión de inventario (\texttt{/admin\_institution/inventory}).
\end{enumerate}

\begin{figure}[H]
    \centering
    \includegraphics[width=0.95\textwidth]{SuperAdminImage/inventario_dashboard_admin.png}
    \caption{Página de gestión de inventario}
    \label{fig:gestion-inventario-inst}
\end{figure}

\subsection{Funcionalidades Disponibles}

\begin{itemize}
    \item \textbf{Ver todos los inventarios:} Visualice inventarios de su institución.
    \item \textbf{Buscar inventarios:} Busque inventarios por nombre.
    \item \textbf{Filtrar inventarios:} Filtre por estado y tipo.
    \item \textbf{Crear inventario:} Cree nuevos inventarios para su institución.
    \item \textbf{Editar inventario:} Modifique información de inventarios existentes.
    \item \textbf{Activar/Desactivar inventario:} Cambie el estado de un inventario.
    \item \textbf{Ver items de inventario:} Acceda a la lista de items de cada inventario.
    \item \textbf{Gestionar items:} Agregue, edite o elimine items de inventarios.
\end{itemize}

\subsection{Crear Nuevo Inventario}

Para crear un nuevo inventario en su institución:

\begin{enumerate}
    \item En la página de gestión de inventario, localice el botón \textbf{"Nuevo Inventario"} o \textbf{"Crear Inventario"}.
\end{enumerate}

\begin{figure}[H]
    \centering
    \includegraphics[width=0.3\textwidth]{SuperAdminImage/boton_crear_inventario_admin.png}
    \caption{Botón para crear nuevo inventario}
    \label{fig:boton-crear-inventario-inst}
\end{figure}

\begin{enumerate}
    \setcounter{enumi}{1}
    \item Haga clic en el botón para abrir el modal de creación de inventario.
    \item Complete el formulario del modal con la siguiente información:
    \begin{itemize}
        \item \textbf{Nombre del inventario:} Ingrese un nombre descriptivo para el inventario.
        \item \textbf{Descripción:} Proporcione una descripción detallada del inventario.
        \item \textbf{Institución:} Se asigna automáticamente su institución.
        \item \textbf{Imagen del inventario:} Opcionalmente, suba una imagen representativa.
        \item \textbf{Estado:} Seleccione si el inventario estará Activo o Inactivo.
        \item \textbf{Usuario responsable:} Asigne un usuario responsable del inventario (debe ser usuario de su institución).
    \end{itemize}
    \item Revise toda la información ingresada.
    \item Haga clic en \textbf{"Guardar"} o \textbf{"Crear"} para confirmar la creación.
\end{enumerate}

\begin{figure}[H]
    \centering
    \includegraphics[width=0.9\textwidth]{SuperAdminImage/nuevo_inventario_modal_admin.png}
    \caption{Modal de creación de inventario con todos los campos del formulario}
    \label{fig:modal-crear-inventario-inst}
\end{figure}

\subsection{Editar Inventario}

Para editar un inventario existente de su institución:

\begin{enumerate}
    \item En la lista de inventarios, localice el inventario que desea editar.
    \item Localice el botón \textbf{"Editar"} o el ícono de edición en la fila del inventario.
\end{enumerate}

\begin{figure}[H]
    \centering
    \includegraphics[width=0.3\textwidth]{SuperAdminImage/boton_editar_usuario_admin.png}
    \caption{Botón de editar inventario en la lista}
    \label{fig:boton-editar-inventario-inst}
\end{figure}

\begin{enumerate}
    \setcounter{enumi}{2}
    \item Haga clic en el botón para abrir el modal de edición.
    \item El modal mostrará el formulario con los datos actuales del inventario.
    \item Modifique los campos necesarios (nombre, descripción, estado, usuario responsable, etc.).
    \item Haga clic en \textbf{"Guardar"} para aplicar los cambios.
\end{enumerate}

\begin{figure}[H]
    \centering
    \includegraphics[width=0.9\textwidth]{SuperAdminImage/editar_inventario_modal_sgdis.png}
    \caption{Modal de edición de inventario con los campos prellenados}
    \label{fig:modal-editar-inventario-inst}
\end{figure}

\subsection{Ver Items de un Inventario}

Para ver los items de un inventario específico:

\begin{enumerate}
    \item En la lista de inventarios, localice el inventario del cual desea ver los items.
    \item Localice el botón \textbf{"Ver Items"} o el ícono de items en la fila del inventario.
\end{enumerate}

\begin{figure}[H]
    \centering
    \includegraphics[width=0.7\textwidth]{SuperAdminImage/ver_items_inventario_admin.png}
    \caption{Botón para ver items de un inventario}
    \label{fig:boton-ver-items-inst}
\end{figure}

\begin{enumerate}
    \setcounter{enumi}{2}
    \item Haga clic en el botón para acceder a la página de items del inventario.
    \item Será redirigido a una página que muestra todos los items del inventario seleccionado.
\end{enumerate}

\begin{figure}[H]
    \centering
    \includegraphics[width=0.95\textwidth]{SuperAdminImage/ver_items_modal_inventario_admin.png}
    \caption{Página de items de un inventario con lista de items}
    \label{fig:pagina-items-inventario-inst}
\end{figure}

\subsection{Asignar Manager a Inventario}

Para asignar un usuario responsable (manager) a un inventario:

\begin{enumerate}
    \item En la lista de inventarios, localice el inventario al cual desea asignar un manager.
    \item Localice el botón \textbf{"Asignar Manager"} o \textbf{"Asignar Responsable"}.
\end{enumerate}

\begin{figure}[H]
    \centering
    \includegraphics[width=0.2\textwidth]{SuperAdminImage/asiganar_manager_inventario.png}
    \caption{Botón para asignar manager a inventario}
    \label{fig:boton-asignar-manager-inst}
\end{figure}

\begin{enumerate}
    \setcounter{enumi}{2}
    \item Haga clic en el botón para abrir el modal de asignación.
    \item El modal mostrará:
    \begin{itemize}
        \item Lista de usuarios disponibles de su institución.
        \item Usuario actual asignado (si existe).
        \item Campo para seleccionar el nuevo usuario responsable.
        \item Opción para asignar un rol específico al usuario en el inventario.
    \end{itemize}
    \item Seleccione el usuario que será el manager del inventario.
    \item Opcionalmente, asigne un rol específico.
    \item Haga clic en \textbf{"Asignar"} o \textbf{"Guardar"} para confirmar.
\end{enumerate}

\begin{figure}[H]
    \centering
    \includegraphics[width=0.6\textwidth]{SuperAdminImage/modal-asignar-manager.png}
    \caption{Modal de asignación de manager con lista de usuarios}
    \label{fig:modal-asignar-manager-inst}
\end{figure}

\subsection{Jerarquía de Inventario}

Para visualizar y gestionar la jerarquía de un inventario:

\begin{enumerate}
    \item En la lista de inventarios, localice el inventario del cual desea ver la jerarquía.
    \item Localice el botón \textbf{"Ver Jerarquía"} o el ícono de estructura jerárquica.
\end{enumerate}

\begin{figure}[H]
    \centering
    \includegraphics[width=0.2\textwidth]{SuperAdminImage/jerarquia_inventarios_admin.png}
    \caption{Botón para ver jerarquía de inventario}
    \label{fig:boton-jerarquia-inventario-inst}
\end{figure}

\begin{enumerate}
    \setcounter{enumi}{2}
    \item Haga clic en el botón para abrir el modal de jerarquía.
    \item El modal mostrará la estructura organizacional del inventario:
    \begin{itemize}
        \item \textbf{Institución:} Institución asociada al inventario (su institución).
        \item \textbf{Usuarios asignados:} Lista de usuarios con acceso, organizados por rol.
        \item \textbf{Sub-inventarios:} Inventarios relacionados o dependientes (si aplica).
        \item \textbf{Diagrama visual:} Representación gráfica de la jerarquía.
    \end{itemize}
    \item Puede expandir o colapsar los diferentes niveles de la jerarquía.
    \item Desde este modal puede acceder a editar roles o usuarios directamente.
\end{enumerate}

\begin{figure}[H]
    \centering
    \includegraphics[width=0.9\textwidth]{SuperAdminImage/modal_jerarquia_admin.png}
    \caption{Modal de jerarquía de inventario con estructura organizacional}
    \label{fig:modal-jerarquia-inventario-inst}
\end{figure}

\subsection{Eliminar Inventario}

Para eliminar un inventario del sistema:

\begin{enumerate}
    \item En la lista de inventarios, localice el inventario que desea eliminar.
    \item Localice el botón \textbf{"Eliminar"} o el ícono de papelera en la fila del inventario.
\end{enumerate}

\begin{figure}[H]
    \centering
    \includegraphics[width=0.2\textwidth]{SuperAdminImage/boton_eliminar.png}
    \caption{Botón para eliminar inventario}
    \label{fig:boton-eliminar-inventario-inst}
\end{figure}

\begin{enumerate}
    \setcounter{enumi}{2}
    \item Haga clic en el botón para abrir el modal de confirmación de eliminación.
    \item El modal mostrará una advertencia indicando:
    \begin{itemize}
        \item Nombre del inventario a eliminar.
        \item Cantidad de items que se eliminarán junto con el inventario.
        \item Transferencias y préstamos asociados que se verán afectados.
        \item Advertencia de que esta acción no se puede deshacer.
    \end{itemize}
    \item Revise cuidadosamente la información antes de confirmar.
    \item Escriba el nombre del inventario para confirmar (medida de seguridad adicional).
    \item Haga clic en \textbf{"Eliminar"} para eliminar el inventario permanentemente.
\end{enumerate}

\begin{figure}[H]
    \centering
    \includegraphics[width=0.7\textwidth]{SuperAdminImage/modal_eliminar_inventario.png}
    \caption{Modal de confirmación de eliminación de inventario}
    \label{fig:modal-eliminar-inventario-inst}
\end{figure}

\subsection{Gestionar Items de un Inventario}

Una vez en la página de items de un inventario, puede realizar las siguientes acciones:

\paragraph{Ver Detalles de un Item}

\begin{enumerate}
    \item En la lista de items, localice el item del cual desea ver los detalles.
    \item Localice el botón \textbf{"Ver"} o el ícono de ojo en la fila del item.
\end{enumerate}

\begin{figure}[H]
    \centering
    \includegraphics[width=0.1\textwidth]{SuperAdminImage/ver_item_admin_icon.png}
    \caption{Botón para ver detalles de un item}
    \label{fig:boton-ver-item-inst}
\end{figure}

\begin{enumerate}
    \setcounter{enumi}{2}
    \item Haga clic en el botón para abrir el modal de detalles del item.
    \item El modal mostrará toda la información del item:
    \begin{itemize}
        \item Información básica (nombre, descripción, placa).
        \item Categoría y estado actual.
        \item Valor y ubicación.
        \item Imágenes asociadas (si las hay).
        \item Historial de transferencias.
        \item Historial de préstamos.
        \item Historial de verificaciones.
    \end{itemize}
    \item Desde este modal puede editar el item haciendo clic en el botón \textbf{"Editar"}.
\end{enumerate}

\begin{figure}[H]
    \centering
    \includegraphics[width=0.9\textwidth]{SuperAdminImage/visualizacion_item_inventario.png}
    \caption{Modal de detalles del item con toda la información}
    \label{fig:modal-ver-item-inst}
\end{figure}

\paragraph{Editar Item}

\begin{enumerate}
    \item En la lista de items, localice el item que desea editar.
    \item Localice el botón \textbf{"Editar"} o el ícono de edición en la fila del item.
\end{enumerate}

\begin{figure}[H]
    \centering
    \includegraphics[width=0.4\textwidth]{SuperAdminImage/boton_editar_usuario_admin.png}
    \caption{Botón para editar item}
    \label{fig:boton-editar-item-inst}
\end{figure}

\begin{enumerate}
    \setcounter{enumi}{2}
    \item Haga clic en el botón para abrir el modal de edición.
    \item El modal mostrará el formulario con los datos actuales del item.
    \item Modifique los campos necesarios (nombre, descripción, estado, valor, ubicación, etc.).
    \item Puede agregar o eliminar imágenes del item.
    \item Haga clic en \textbf{"Guardar"} para aplicar los cambios.
\end{enumerate}

\begin{figure}[H]
    \centering
    \includegraphics[width=0.9\textwidth]{SuperAdminImage/modal_editar_item_admin.png}
    \caption{Modal de edición de item con los campos prellenados}
    \label{fig:modal-editar-item-inst}
\end{figure}

\paragraph{Eliminar Item}

\begin{enumerate}
    \item En la lista de items, localice el item que desea eliminar.
    \item Localice el botón \textbf{"Eliminar"} o el ícono de eliminar (papelera) en la fila del item.
\end{enumerate}

\begin{figure}[H]
    \centering
    \includegraphics[width=0.2\textwidth]{SuperAdminImage/boton_eliminar.png}
    \caption{Botón para eliminar item}
    \label{fig:boton-eliminar-item-inst}
\end{figure}

\begin{enumerate}
    \setcounter{enumi}{2}
    \item Haga clic en el botón para abrir el modal de confirmación de eliminación.
    \item El modal mostrará una advertencia indicando que esta acción no se puede deshacer.
    \item Revise el mensaje de confirmación.
    \item Haga clic en \textbf{"Eliminar"} o \textbf{"Confirmar"} para eliminar el item permanentemente.
\end{enumerate}

\begin{figure}[H]
    \centering
    \includegraphics[width=0.7\textwidth]{SuperAdminImage/alerta_eliminar_item.png}
    \caption{Modal de confirmación de eliminación de item}
    \label{fig:modal-eliminar-item-inst}
\end{figure}

\paragraph{Prestar Item desde Inventario}

Para crear un préstamo directamente desde la vista de items:

\begin{enumerate}
    \item En la lista de items del inventario, localice el item que desea prestar.
    \item Localice el botón \textbf{"Prestar"} o el ícono de préstamo en la fila del item.
\end{enumerate}

\begin{figure}[H]
    \centering
    \includegraphics[width=0.1\textwidth]{SuperAdminImage/prestar_items_admin.png}  
    \caption{Botón para prestar item desde inventario}
    \label{fig:boton-prestar-item-inv-inst}
\end{figure}

\begin{enumerate}
    \setcounter{enumi}{2}
    \item Haga clic en el botón para abrir el modal de préstamo.
    \item El modal mostrará la información del item y solicitará:
    \begin{itemize}
        \item \textbf{Responsable del préstamo:} Usuario que recibirá el item en préstamo (debe ser usuario de su institución).
        \item \textbf{Fecha de devolución estimada:} Fecha prevista para la devolución.
        \item \textbf{Motivo del préstamo:} Razón por la cual se solicita el préstamo.
        \item \textbf{Observaciones:} Comentarios adicionales (opcional).
    \end{itemize}
    \item Complete la información requerida.
    \item Haga clic en \textbf{"Prestar Item"} para confirmar el préstamo.
\end{enumerate}

\begin{figure}[H]
    \centering
    \includegraphics[width=0.6\textwidth]{SuperAdminImage/modal_prestar_items_admin.png}
    \caption{Modal de préstamo de item desde inventario}
    \label{fig:modal-prestar-item-inv-inst}
\end{figure}

\paragraph{Transferir Item}

Para transferir un item a otro inventario:

\begin{enumerate}
    \item En la lista de items del inventario, localice el item que desea transferir.
    \item Localice el botón \textbf{"Transferir"} o el ícono de transferencia en la fila del item.
\end{enumerate}

\begin{figure}[H]
    \centering
    \includegraphics[width=0.1\textwidth]{SuperAdminImage/transferir_item_admin.png}
    \caption{Botón para transferir item}
    \label{fig:boton-transferir-item-inst}
\end{figure}

\begin{enumerate}
    \setcounter{enumi}{2}
    \item Haga clic en el botón para abrir el modal de transferencia.
    \item El modal mostrará la información del item y solicitará:
    \begin{itemize}
        \item \textbf{Inventario destino:} Seleccione el inventario al cual transferir el item.
        \item \textbf{Motivo de transferencia:} Razón por la cual se realiza la transferencia.
        \item \textbf{Comentarios adicionales:} Observaciones sobre la transferencia (opcional).
    \end{itemize}
    \item Complete la información requerida.
    \item Haga clic en \textbf{"Solicitar Transferencia"} para enviar la solicitud.
\end{enumerate}

\textbf{Nota:} Las transferencias a inventarios fuera de su institución requerirán aprobación adicional.

\paragraph{Historial de Transferencias de Item}

Para ver el historial completo de transferencias de un item:

\begin{enumerate}
    \item En la lista de items, localice el item del cual desea ver el historial de transferencias.
    \item Localice el botón \textbf{"Historial Transferencias"} o el ícono de historial.
\end{enumerate}

\begin{figure}[H]
    \centering
    \includegraphics[width=0.1\textwidth]{SuperAdminImage/historial_transferencia.png}
    \caption{Botón para ver historial de transferencias}
    \label{fig:boton-historial-trans-inst}
\end{figure}

\begin{enumerate}
    \setcounter{enumi}{2}
    \item Haga clic en el botón para abrir el modal de historial.
    \item El modal mostrará el historial completo de transferencias del item:
    \begin{itemize}
        \item \textbf{Línea de tiempo:} Visualización cronológica de todas las transferencias.
        \item \textbf{Inventarios anteriores:} Lista de inventarios donde ha estado el item.
        \item \textbf{Fechas de transferencia:} Cuándo se realizó cada transferencia.
        \item \textbf{Usuarios involucrados:} Quién solicitó y aprobó cada transferencia.
        \item \textbf{Estados:} Estado de cada transferencia (Aprobada, Rechazada, Pendiente).
    \end{itemize}
    \item Puede filtrar el historial por fecha o estado.
    \item Haga clic en cualquier entrada para ver más detalles.
\end{enumerate}

\begin{figure}[H]
    \centering
    \includegraphics[width=0.6\textwidth]{SuperAdminImage/modal_historial_transferencia.png}
    \caption{Modal de historial de transferencias del item}
    \label{fig:modal-historial-trans-inst}
\end{figure}

\paragraph{Solicitar Cancelación de Item}

Para solicitar la cancelación (baja) de un item:

\begin{enumerate}
    \item En la lista de items, localice el item que desea solicitar para cancelación.
    \item Localice el botón \textbf{"Solicitar Cancelación"} o el ícono correspondiente.
\end{enumerate}

\begin{figure}[H]
    \centering
    \includegraphics[width=0.1\textwidth]{SuperAdminImage/boton_solicitar_cancelacion.png}
    \caption{Botón para solicitar cancelación de item}
    \label{fig:boton-solicitar-cancelacion-inst}
\end{figure}

\begin{enumerate}
    \setcounter{enumi}{2}
    \item Haga clic en el botón para abrir el modal de solicitud de cancelación.
    \item El modal mostrará la información del item y solicitará:
    \begin{itemize}
        \item \textbf{Motivo de cancelación:} Seleccione el motivo (Daño irreparable, Obsolescencia, Pérdida, etc.).
        \item \textbf{Descripción detallada:} Explique las razones de la solicitud de cancelación.
        \item \textbf{Evidencia:} Opcionalmente, adjunte fotos o documentos que respalden la solicitud.
        \item \textbf{Valor residual:} Indique el valor residual del item si aplica.
    \end{itemize}
    \item Complete la información requerida.
    \item Haga clic en \textbf{"Enviar Solicitud"} para enviar la solicitud de cancelación.
\end{enumerate}

\begin{figure}[H]
    \centering
    \includegraphics[width=0.4\textwidth]{SuperAdminImage/solicitar_cancelacion.png}
    \caption{Modal de solicitud de cancelación de item}
    \label{fig:modal-solicitar-cancelacion-inst}
\end{figure}

\subsection{Notas Importantes sobre Gestión de Inventario}

\begin{itemize}
    \item Solo puede gestionar inventarios que pertenezcan a su institución.
    \item Todos los inventarios creados pertenecerán automáticamente a su institución.
    \item Los usuarios responsables que asigne deben ser usuarios de su institución.
    \item Las transferencias a inventarios de otras instituciones o regionales requerirán aprobación.
\end{itemize}

\section{Gestión de Transferencias}

Administre todas las transferencias de items que involucren inventarios de su institución.

\subsection{Acceso a Gestión de Transferencias}

\begin{enumerate}
    \item En el panel lateral, haga clic en \textbf{"Transferencias"}.
    \item Será redirigido a la página de gestión de transferencias (\texttt{/admin\_institution/transfers}).
\end{enumerate}

\begin{figure}[H]
    \centering
    \includegraphics[width=0.95\textwidth]{SuperAdminImage/gestion_transferencias_admin.png}
    \caption{Página de gestión de transferencias}
    \label{fig:gestion-transferencias-inst}
\end{figure}

\subsection{Funcionalidades Disponibles}

\begin{itemize}
    \item \textbf{Ver todas las transferencias:} Visualice transferencias que involucren inventarios de su institución.
    \item \textbf{Filtrar transferencias:} Filtre por estado, inventario origen, inventario destino, fecha, etc.
    \item \textbf{Aprobar transferencias:} Apruebe transferencias pendientes dentro de su institución.
    \item \textbf{Rechazar transferencias:} Rechace transferencias con justificación.
    \item \textbf{Ver detalles de transferencia:} Acceda a información detallada de cada transferencia.
    \item \textbf{Crear nueva transferencia:} Cree solicitudes de transferencia.
\end{itemize}

\subsection{Crear Nueva Transferencia}

Para crear una nueva solicitud de transferencia:

\begin{enumerate}
    \item En la página de gestión de transferencias, localice el botón \textbf{"Nueva Transferencia"} en la parte superior.
\end{enumerate}

\begin{figure}[H]
    \centering
    \includegraphics[width=0.3\textwidth]{SuperAdminImage/boton_nueva_transferencia.png}
    \caption{Botón para crear nueva transferencia}
    \label{fig:boton-nueva-transferencia-inst}
\end{figure}

\begin{enumerate}
    \setcounter{enumi}{1}
    \item Haga clic en el botón para abrir el modal de nueva transferencia.
    \item El modal solicitará la siguiente información:
    \begin{itemize}
        \item \textbf{Inventario origen:} Seleccione el inventario desde donde se transferirá el item (debe ser de su institución).
        \item \textbf{Item a transferir:} Seleccione el item de la lista del inventario origen.
        \item \textbf{Inventario destino:} Seleccione el inventario destino.
        \item \textbf{Motivo de la transferencia:} Explique por qué se realiza la transferencia.
        \item \textbf{Fecha deseada:} Fecha tentativa para realizar la transferencia.
        \item \textbf{Comentarios adicionales:} Observaciones opcionales.
    \end{itemize}
    \item Revise la información ingresada.
    \item Haga clic en \textbf{"Crear Transferencia"} para enviar la solicitud.
\end{enumerate}

\begin{figure}[H]
    \centering
    \includegraphics[width=0.6\textwidth]{SuperAdminImage/modal_nueva_transferencia.png}
    \caption{Modal de creación de nueva transferencia}
    \label{fig:modal-nueva-transferencia-inst}
\end{figure}

\subsection{Ver Detalles de Transferencia}

Para ver los detalles completos de una transferencia:

\begin{enumerate}
    \item En la lista de transferencias, localice la transferencia de la cual desea ver los detalles.
    \item Localice el botón \textbf{"Ver Detalles"} o el ícono de ojo en la fila de la transferencia.
\end{enumerate}

\begin{figure}[H]
    \centering
    \includegraphics[width=0.2\textwidth]{SuperAdminImage/ver_item_admin_icon.png}
    \caption{Botón para ver detalles de transferencia}
    \label{fig:boton-ver-transferencia-inst}
\end{figure}

\begin{enumerate}
    \setcounter{enumi}{2}
    \item Haga clic en el botón para abrir el modal de detalles.
    \item El modal mostrará información completa de la transferencia:
    \begin{itemize}
        \item \textbf{Información del item:} Nombre, placa, descripción, imágenes.
        \item \textbf{Inventario origen:} Nombre y ubicación del inventario de origen.
        \item \textbf{Inventario destino:} Nombre y ubicación del inventario destino.
        \item \textbf{Usuario solicitante:} Información del usuario que solicitó la transferencia.
        \item \textbf{Fecha de solicitud:} Fecha y hora en que se realizó la solicitud.
        \item \textbf{Estado actual:} Estado de la transferencia (Pendiente, Aprobada, Rechazada).
        \item \textbf{Motivo:} Motivo de la transferencia.
        \item \textbf{Comentarios:} Comentarios adicionales si los hay.
        \item \textbf{Historial:} Historial de cambios de estado.
    \end{itemize}
    \item Desde este modal puede aprobar o rechazar la transferencia si está pendiente y tiene permisos.
\end{enumerate}

\begin{figure}[H]
    \centering
    \includegraphics[width=0.9\textwidth]{SuperAdminImage/detalles_transferencia_admin.png}
    \caption{Modal de detalles de transferencia con toda la información}
    \label{fig:modal-ver-transferencia-inst}
\end{figure}

\subsection{Aprobar Transferencia}

Para aprobar una transferencia pendiente dentro de su institución:

\begin{enumerate}
    \item En la lista de transferencias, localice la transferencia con estado \textbf{"Pendiente"}.
    \item Localice el botón \textbf{"Aprobar"} o el ícono de aprobación.
\end{enumerate}

\begin{figure}[H]
    \centering
    \includegraphics[width=0.1\textwidth]{SuperAdminImage/apro_transferencia_admin.png}
    \caption{Botón para aprobar transferencia}
    \label{fig:boton-aprobar-transferencia-inst}
\end{figure}

\begin{enumerate}
    \setcounter{enumi}{2}
    \item Haga clic en el botón para abrir el modal de aprobación.
    \item El modal mostrará los detalles de la transferencia.
    \item Revise toda la información cuidadosamente.
    \item Opcionalmente, puede agregar comentarios o notas.
    \item Haga clic en \textbf{"Aprobar"} o \textbf{"Confirmar"} para aprobar la transferencia.
\end{enumerate}

\begin{figure}[H]
    \centering
    \includegraphics[width=0.85\textwidth]{SuperAdminImage/modal_transderencias_aprobar.png}
    \caption{Modal de aprobación de transferencia con los detalles}
    \label{fig:modal-aprobar-transferencia-inst}
\end{figure}

\subsection{Rechazar Transferencia}

Para rechazar una transferencia pendiente:

\begin{enumerate}
    \item En la lista de transferencias, localice la transferencia con estado \textbf{"Pendiente"}.
    \item Localice el botón \textbf{"Rechazar"} o el ícono de rechazo en la fila de la transferencia.
\end{enumerate}

\begin{figure}[H]
    \centering
    \includegraphics[width=0.2\textwidth]{SuperAdminImage/cancelar_transferencia_icon.png}
    \caption{Botón para rechazar transferencia}
    \label{fig:boton-rechazar-transferencia-inst}
\end{figure}

\begin{enumerate}
    \setcounter{enumi}{2}
    \item Haga clic en el botón para abrir el modal de rechazo.
    \item El modal mostrará los detalles de la transferencia y un campo para justificación.
    \item Ingrese el motivo del rechazo en el campo \textbf{"Justificación"} (obligatorio).
    \item Revise la información antes de confirmar.
    \item Haga clic en \textbf{"Rechazar"} o \textbf{"Confirmar"} para rechazar la transferencia.
\end{enumerate}

\begin{figure}[H]
    \centering
    \includegraphics[width=0.85\textwidth]{SuperAdminImage/modal_rechazar_transferencia.png}
    \caption{Modal de rechazo de transferencia con campo de justificación}
    \label{fig:modal-rechazar-transferencia-inst}
\end{figure}

\section{Gestión de Verificaciones}

Administre las verificaciones de items realizadas en inventarios de su institución.

\subsection{Acceso a Gestión de Verificaciones}

\begin{enumerate}
    \item En el panel lateral, haga clic en \textbf{"Verificación"}.
    \item Será redirigido a la página de gestión de verificaciones (\texttt{/admin\_institution/verification}).
\end{enumerate}

\begin{figure}[H]
    \centering
    \includegraphics[width=0.95\textwidth]{SuperAdminImage/gestion_verificaciones_admin .png}
    \caption{Página de gestión de verificaciones}
    \label{fig:gestion-verificaciones-inst}
\end{figure}

\subsection{Funcionalidades Disponibles}

\begin{itemize}
    \item \textbf{Ver todas las verificaciones:} Visualice verificaciones de todos los inventarios de su institución.
    \item \textbf{Filtrar verificaciones:} Filtre por inventario, item, fecha, usuario, etc.
    \item \textbf{Ver detalles de verificación:} Acceda a información completa de cada verificación.
    \item \textbf{Ver imágenes de verificación:} Visualice las fotografías asociadas a las verificaciones.
    \item \textbf{Crear nueva verificación:} Cree nuevas verificaciones de items.
    \item \textbf{Exportar verificaciones:} Exporte datos de verificaciones para análisis.
\end{itemize}

\subsection{Nueva Verificación}

Para crear una nueva verificación de item:

\begin{enumerate}
    \item En la página de gestión de verificaciones, localice el botón \textbf{"Nueva Verificación"}.
\end{enumerate}

\begin{figure}[H]
    \centering
    \includegraphics[width=0.3\textwidth]{SuperAdminImage/boton_nueva_transferencia.png}
    \caption{Botón para crear nueva verificación}
    \label{fig:boton-nueva-verificacion-inst}
\end{figure}

\begin{enumerate}
    \setcounter{enumi}{1}
    \item Haga clic en el botón para abrir el modal de nueva verificación.
    \item El modal solicitará la siguiente información:
    \begin{itemize}
        \item \textbf{Inventario:} Seleccione el inventario donde se encuentra el item (debe ser de su institución).
        \item \textbf{Item a verificar:} Seleccione el item de la lista.
        \item \textbf{Estado observado:} Seleccione el estado actual del item (Buen estado, Dañado, etc.).
        \item \textbf{Ubicación actual:} Confirme o actualice la ubicación del item.
        \item \textbf{Observaciones:} Ingrese observaciones detalladas sobre la verificación.
        \item \textbf{Evidencia fotográfica:} Adjunte fotos del item verificado.
    \end{itemize}
    \item Complete la información requerida.
    \item Haga clic en \textbf{"Guardar Verificación"} para confirmar.
\end{enumerate}

\begin{figure}[H]
    \centering
    \includegraphics[width=0.6\textwidth]{SuperAdminImage/modal_nueva_verificacion.png}
    \caption{Modal de creación de nueva verificación}
    \label{fig:modal-nueva-verificacion-inst}
\end{figure}

\subsection{Ver Detalles de Verificación}

Para ver los detalles completos de una verificación:

\begin{enumerate}
    \item En la lista de verificaciones, localice la verificación de la cual desea ver los detalles.
    \item Localice el botón \textbf{"Ver Detalles"} o el ícono de ojo en la fila de la verificación.
\end{enumerate}

\begin{figure}[H]
    \centering
    \includegraphics[width=0.2\textwidth]{SuperAdminImage/ver_item_admin_icon.png}
    \caption{Botón para ver detalles de verificación}
    \label{fig:boton-ver-verificacion-inst}
\end{figure}

\begin{enumerate}
    \setcounter{enumi}{2}
    \item Haga clic en el botón para abrir el modal de detalles.
    \item El modal mostrará información completa de la verificación:
    \begin{itemize}
        \item \textbf{Información del item:} Nombre, placa, descripción del item verificado.
        \item \textbf{Inventario:} Inventario al que pertenece el item.
        \item \textbf{Usuario verificador:} Información del usuario que realizó la verificación.
        \item \textbf{Fecha de verificación:} Fecha y hora en que se realizó la verificación.
        \item \textbf{Estado del item:} Estado observado durante la verificación.
        \item \textbf{Observaciones:} Observaciones y notas del verificador.
        \item \textbf{Imágenes:} Galería de imágenes tomadas durante la verificación.
        \item \textbf{Ubicación:} Ubicación donde se encontró el item.
    \end{itemize}
    \item Puede navegar entre las imágenes usando los controles del modal.
    \item Puede descargar las imágenes individualmente o en conjunto.
\end{enumerate}

\begin{figure}[H]
    \centering
    \includegraphics[width=0.8\textwidth]{SuperAdminImage/verificacion_MODAL_ADMIN.png}
    \caption{Modal de detalles de verificación con imágenes y toda la información}
    \label{fig:modal-ver-verificacion-inst}
\end{figure}

\section{Gestión de Préstamos}

Administre todos los préstamos de items de inventarios de su institución.

\subsection{Acceso a Gestión de Préstamos}

\begin{enumerate}
    \item En el panel lateral, haga clic en \textbf{"Préstamos"}.
    \item Será redirigido a la página de gestión de préstamos (\texttt{/admin\_institution/loans}).
\end{enumerate}

\begin{figure}[H]
    \centering
    \includegraphics[width=0.95\textwidth]{SuperAdminImage/gestion_prestamos_dashboard_admin.png}
    \caption{Página de gestión de préstamos}
    \label{fig:gestion-prestamos-inst}
\end{figure}

\subsection{Funcionalidades Disponibles}

\begin{itemize}
    \item \textbf{Ver todos los préstamos:} Visualice préstamos de todos los inventarios de su institución.
    \item \textbf{Filtrar préstamos:} Filtre por estado, inventario, usuario responsable, fecha, etc.
    \item \textbf{Crear préstamo:} Registre nuevos préstamos directamente.
    \item \textbf{Registrar devolución:} Registre la devolución de items prestados.
    \item \textbf{Ver historial de préstamos:} Revise el historial completo de préstamos.
\end{itemize}

\subsection{Crear Préstamo}

Para registrar un nuevo préstamo en el sistema:

\begin{enumerate}
    \item Desde la sección \textbf{Gestión de Préstamos}, ubique en la parte superior derecha el botón \textbf{"Prestar Ítem"}.
\end{enumerate}

\begin{figure}[H]
    \centering
    \includegraphics[width=0.3\textwidth]{SuperAdminImage/prestar_items_boton.png}
    \caption{Botón para crear un nuevo préstamo}
    \label{fig:boton-prestar-item-inst}
\end{figure}

\begin{enumerate}
    \setcounter{enumi}{1}
    \item Al hacer clic en el botón, se abrirá el modal de registro de préstamo.
    \item El formulario solicitará la siguiente información:
    \begin{itemize}
        \item \textbf{Institución:} Se autocompleta con su institución.
        \item \textbf{Inventario:} Indique el inventario en el que está registrado el ítem.
        \item \textbf{Ítem:} Seleccione el ítem que será prestado.
        \item \textbf{Responsable:} Persona a cargo del préstamo (debe ser usuario de su institución).
        \item \textbf{Detalles (Opcional):} Comentarios adicionales relacionados con el préstamo.
    \end{itemize}
    \item Revise cuidadosamente la información ingresada.
    \item Finalmente, haga clic en \textbf{"Prestar ítem"} para confirmar la creación del préstamo.
\end{enumerate}

\begin{figure}[H]
    \centering
    \includegraphics[width=0.85\textwidth]{SuperAdminImage/modal_prestar_item.png}
    \caption{Modal para registrar un préstamo}
    \label{fig:modal-prestar-item-inst}
\end{figure}

Una vez completado el proceso, el préstamo aparecerá en la tabla principal con estado \textbf{"Prestado"}, mostrando la fecha y hora en que fue registrado.

\subsection{Registrar Devolución}

Para registrar la devolución de un item prestado:

\begin{enumerate}
    \item En la lista de préstamos, localice el préstamo con estado \textbf{"Prestado"}.
    \item Localice el botón \textbf{"Registrar Devolución"} o el ícono correspondiente.
\end{enumerate}

\begin{figure}[H]
    \centering
    \includegraphics[width=0.2\textwidth]{SuperAdminImage/devolver_prestamo_admin.png}
    \caption{Botón para registrar devolución}
    \label{fig:boton-devolucion-prestamo-inst}
\end{figure}

\begin{enumerate}
    \setcounter{enumi}{2}
    \item Haga clic en el botón para abrir el modal de devolución.
    \item El modal mostrará los detalles del préstamo y solicitará:
    \begin{itemize}
        \item Confirmación de la devolución.
        \item Estado del item (en buen estado, dañado, etc.).
        \item Observaciones sobre la devolución (opcional).
    \end{itemize}
    \item Complete la información requerida.
    \item Haga clic en \textbf{"Confirmar Devolución"} para registrar la devolución.
\end{enumerate}

\begin{figure}[H]
    \centering
    \includegraphics[width=0.85\textwidth]{SuperAdminImage/modal_devolver_prestamo.png}
    \caption{Modal de registro de devolución de préstamo}
    \label{fig:modal-devolucion-prestamo-inst}
\end{figure}

\section{Gestión de Cancelaciones (Bajas)}

Administre las cancelaciones de items de inventarios de su institución.

\subsection{Acceso a Gestión de Cancelaciones}

\begin{enumerate}
    \item En el panel lateral, haga clic en \textbf{"Bajas"}.
    \item Será redirigido a la página de gestión de cancelaciones (\texttt{/admin\_institution/cancellations}).
\end{enumerate}

\begin{figure}[H]
    \centering
    \includegraphics[width=0.95\textwidth]{SuperAdminImage/gestion_cancelaciones_dashboard.png}
    \caption{Página de gestión de cancelaciones}
    \label{fig:gestion-cancelaciones-inst}
\end{figure}

\subsection{Funcionalidades Disponibles}

\begin{itemize}
    \item \textbf{Ver todas las cancelaciones:} Visualice cancelaciones de todos los inventarios de su institución.
    \item \textbf{Filtrar cancelaciones:} Filtre por inventario, item, fecha, motivo, etc.
    \item \textbf{Aprobar cancelaciones:} Apruebe solicitudes de cancelación pendientes.
    \item \textbf{Rechazar cancelaciones:} Rechace cancelaciones con justificación.
    \item \textbf{Ver detalles de cancelación:} Acceda a información completa de cada cancelación.
\end{itemize}

\subsection{Aprobar Cancelación}

Para aprobar una solicitud de cancelación:

\begin{enumerate}
    \item En la lista de cancelaciones, localice la cancelación con estado \textbf{"Pendiente"}.
    \item Localice el botón \textbf{"Aprobar"} o el ícono de aprobación.
\end{enumerate}

\begin{figure}[H]
    \centering
    \includegraphics[width=0.3\textwidth]{SuperAdminImage/boton_aprobar_cancelacion_dash.png}
    \caption{Botón para aprobar cancelación}
    \label{fig:boton-aprobar-cancelacion-inst}
\end{figure}

\begin{enumerate}
    \setcounter{enumi}{2}
    \item Haga clic en el botón para abrir el modal de aprobación.
    \item El modal mostrará los detalles de la cancelación:
    \begin{itemize}
        \item Información del item a cancelar.
        \item Motivo de la cancelación.
        \item Usuario que solicita la cancelación.
        \item Fecha de solicitud.
        \item Inventario al que pertenece el item.
    \end{itemize}
    \item Revise cuidadosamente la información antes de aprobar.
    \item Haga clic en \textbf{"Aprobar"} para confirmar la cancelación.
\end{enumerate}

\begin{figure}[H]
    \centering
    \includegraphics[width=0.85\textwidth]{SuperAdminImage/modal_aprobar_cancelacion_dash.png}
    \caption{Modal de aprobación de cancelación con los detalles}
    \label{fig:modal-aprobar-cancelacion-inst}
\end{figure}

\subsection{Rechazar Cancelación}

Para rechazar una solicitud de cancelación:

\begin{enumerate}
    \item En la lista de cancelaciones, localice la cancelación con estado \textbf{"Pendiente"}.
    \item Localice el botón \textbf{"Rechazar"} o el ícono de rechazo.
\end{enumerate}

\begin{figure}[H]
    \centering
    \includegraphics[width=0.3\textwidth]{SuperAdminImage/boton_cancelar_cancelaciones_dash.png}
    \caption{Botón para rechazar cancelación}
    \label{fig:boton-rechazar-cancelacion-inst}
\end{figure}

\begin{enumerate}
    \setcounter{enumi}{2}
    \item Haga clic en el botón para abrir el modal de rechazo.
    \item El modal mostrará los detalles y solicitará una justificación del rechazo.
    \item Ingrese el motivo del rechazo en el campo \textbf{"Justificación"} (obligatorio).
    \item Haga clic en \textbf{"Rechazar"} para confirmar el rechazo.
\end{enumerate}

\begin{figure}[H]
    \centering
    \includegraphics[width=0.85\textwidth]{SuperAdminImage/modal_rechazar_cancelacion_dash.png}
    \caption{Modal de rechazo de cancelación con campo de justificación}
    \label{fig:modal-rechazar-cancelacion-inst}
\end{figure}

\section{Reportes}

Genere y visualice reportes del sistema para su institución.

\subsection{Acceso a Reportes}

\begin{enumerate}
    \item En el panel lateral, haga clic en \textbf{"Reportes"}.
    \item Será redirigido a la página de reportes (\texttt{/admin\_institution/reports}).
\end{enumerate}

\begin{figure}[H]
    \centering
    \includegraphics[width=0.95\textwidth]{SuperAdminImage/reportes_dashboard_admin.png}
    \caption{Página de generación de reportes}
    \label{fig:gestion-reportes-inst}
\end{figure}

\subsection{Funcionalidades Disponibles}

\begin{itemize}
    \item \textbf{Generar reportes de inventario:} Cree reportes de inventarios de su institución.
    \item \textbf{Generar reportes de transferencias:} Genere reportes de transferencias con diferentes filtros.
    \item \textbf{Generar reportes de verificaciones:} Cree reportes de verificaciones realizadas en su institución.
    \item \textbf{Generar reportes de préstamos:} Genere reportes de préstamos y devoluciones.
    \item \textbf{Exportar reportes:} Exporte reportes en diferentes formatos (PDF, Excel, CSV).
    \item \textbf{Reportes personalizados:} Configure reportes con filtros y parámetros específicos.
\end{itemize}

\section{Auditoría}

Revise el registro de actividades del sistema en su institución.

\subsection{Acceso a Auditoría}

\begin{enumerate}
    \item En el panel lateral, haga clic en \textbf{"Auditoría"}.
    \item Será redirigido a la página de auditoría (\texttt{/admin\_institution/auditory}).
\end{enumerate}

\begin{figure}[H]
    \centering
    \includegraphics[width=0.95\textwidth]{SuperAdminImage/auditoria_dashboard_admin.png}
    \caption{Página de auditoría del sistema}
    \label{fig:gestion-auditoria-inst}
\end{figure}

\subsection{Funcionalidades Disponibles}

\begin{itemize}
    \item \textbf{Ver registro de actividades:} Visualice todas las acciones realizadas en el sistema relacionadas con su institución.
    \item \textbf{Filtrar actividades:} Filtre por usuario, fecha, tipo de acción, módulo, etc.
    \item \textbf{Buscar actividades:} Busque actividades específicas en el registro.
    \item \textbf{Exportar auditoría:} Exporte registros de auditoría para análisis externo.
    \item \textbf{Ver detalles de actividad:} Acceda a información detallada de cada acción registrada.
\end{itemize}

\section{Notificaciones}

Gestione las notificaciones del sistema.

\subsection{Acceso a Notificaciones}

\begin{enumerate}
    \item En el panel lateral, haga clic en \textbf{"Notificaciones"}.
    \item Será redirigido a la página de notificaciones (\texttt{/admin\_institution/notifications}).
\end{enumerate}

\begin{figure}[H]
    \centering
    \includegraphics[width=0.95\textwidth]{SuperAdminImage/notificaciones_dashboard_admin.png}
    \caption{Página de gestión de notificaciones}
    \label{fig:gestion-notificaciones-inst}
\end{figure}

\subsection{Funcionalidades Disponibles}

\begin{itemize}
    \item \textbf{Ver todas las notificaciones:} Visualice todas las notificaciones del sistema relacionadas con su institución.
    \item \textbf{Filtrar notificaciones:} Filtre por tipo, estado (leída/no leída), fecha, etc.
    \item \textbf{Marcar como leída:} Marque notificaciones como leídas.
    \item \textbf{Crear notificación:} Envíe notificaciones a usuarios específicos de su institución.
    \item \textbf{Configurar notificaciones:} Configure preferencias de notificaciones.
\end{itemize}

\section{Importar/Exportar}

Realice operaciones de importación y exportación de datos de su institución.

\subsection{Acceso a Importar/Exportar}

\begin{enumerate}
    \item En el panel lateral, haga clic en \textbf{"Importar/Exportar"}.
    \item Será redirigido a la página de importar/exportar (\texttt{/admin\_institution/import-export}).
\end{enumerate}

\begin{figure}[H]
    \centering
    \includegraphics[width=0.95\textwidth]{SuperAdminImage/importar_exportar_dashboar.png}
    \caption{Página de importar/exportar datos}
    \label{fig:gestion-importar-exportar-inst}
\end{figure}

\subsection{Funcionalidades Disponibles}

\begin{itemize}
    \item \textbf{Importar inventarios:} Importe inventarios desde archivos Excel o CSV (solo para su institución).
    \item \textbf{Importar items:} Importe items masivamente desde archivos.
    \item \textbf{Importar usuarios:} Importe usuarios desde archivos de datos (solo para su institución).
    \item \textbf{Exportar inventarios:} Exporte inventarios a diferentes formatos.
    \item \textbf{Exportar items:} Exporte items a archivos Excel o CSV.
    \item \textbf{Exportar usuarios:} Exporte listas de usuarios de su institución.
    \item \textbf{Plantillas de importación:} Descargue plantillas para importación de datos.
\end{itemize}

\subsection{Importar Datos}

Para importar datos desde un archivo:

\begin{enumerate}
    \item En la página de importar/exportar, localice la sección de \textbf{"Importar"}.
    \item Seleccione el tipo de datos a importar (Inventarios, Items, Usuarios).
    \item Localice el botón \textbf{"Seleccionar Archivo"} o \textbf{"Cargar Archivo"}.
\end{enumerate}

\begin{figure}[H]
    \centering
    \includegraphics[width=0.3\textwidth]{SuperAdminImage/boton_importar_items_dash.png}
    \caption{Botón para seleccionar archivo a importar}
    \label{fig:boton-importar-inst}
\end{figure}

\begin{enumerate}
    \setcounter{enumi}{3}
    \item Haga clic en el botón y seleccione el archivo desde su computadora.
    \item El sistema mostrará un modal de confirmación con:
    \begin{itemize}
        \item Vista previa de los datos a importar.
        \item Cantidad de registros que se importarán.
        \item Validaciones detectadas (errores o advertencias).
        \item Opción para confirmar o cancelar la importación.
    \end{itemize}
    \item Revise la información en el modal.
    \item Haga clic en \textbf{"Confirmar Importación"} para proceder.
\end{enumerate}

\begin{figure}[H]
    \centering
    \includegraphics[width=0.9\textwidth]{SuperAdminImage/importar_excel_cargado.png}
    \caption{Modal de confirmación de importación con vista previa}
    \label{fig:modal-importar-inst}
\end{figure}

\subsection{Exportar Datos}

Para exportar datos del sistema:

\begin{enumerate}
    \item En la página de importar/exportar, localice la sección de \textbf{"Exportar"}.
    \item Seleccione el tipo de datos a exportar (Inventarios, Items, Usuarios, etc.).
    \item Aplique filtros si desea exportar datos específicos de su institución.
    \item Localice el botón \textbf{"Exportar"} o \textbf{"Descargar"}.
\end{enumerate}

\begin{figure}[H]
    \centering
    \includegraphics[width=0.3\textwidth]{SuperAdminImage/boton_exportar_item.png}
    \caption{Botón para exportar datos}
    \label{fig:boton-exportar-inst}
\end{figure}

\begin{enumerate}
    \setcounter{enumi}{4}
    \item Seleccione el formato de exportación (Excel, CSV, PDF).
    \item Haga clic en el botón para iniciar la exportación.
    \item El sistema generará el archivo y lo descargará automáticamente.
\end{enumerate}

\section{Configuración}

Configure parámetros generales del sistema.

\subsection{Acceso a Configuración}

\begin{enumerate}
    \item En el panel lateral, haga clic en \textbf{"Configuración"}.
    \item Será redirigido a la página de configuración (\texttt{/admin\_institution/settings}).
\end{enumerate}

\begin{figure}[H]
    \centering
    \includegraphics[width=0.95\textwidth]{SuperAdminImage/configuracion_dashboard.png}
    \caption{Página de configuración del sistema}
    \label{fig:gestion-configuracion-inst}
\end{figure}

\subsection{Funcionalidades Disponibles}

\begin{itemize}
    \item \textbf{Configuración de apariencia:} Configure la apariencia de todo el sistema, si el usuario desea modo oscuro o claro.
\end{itemize}

\section{Mi Perfil}

Gestione su información personal y configuración de cuenta.

\subsection{Acceso a Mi Perfil}

\begin{enumerate}
    \item En el panel lateral, haga clic en \textbf{"Mi Perfil"}.
    \item Será redirigido a la página de perfil (\texttt{/admin\_institution/info-me}).
\end{enumerate}

\begin{figure}[H]
    \centering
    \includegraphics[width=0.9\textwidth]{SuperAdminImage/user_me_dashboard.png}
    \caption{Página de mi perfil}
    \label{fig:mi-perfil-inst}
\end{figure}

\subsection{Funcionalidades Disponibles}

\begin{itemize}
    \item \textbf{Ver información personal:} Visualice su información de usuario.
    \item \textbf{Cambiar foto de perfil:} Actualice su foto de perfil.
    \item \textbf{Ver mis préstamos:} Revise los préstamos que ha realizado.
    \item \textbf{Cambiar contraseña:} Actualice su contraseña de acceso.
\end{itemize}

\section{Cerrar Sesión}

Para cerrar sesión en el sistema:

\begin{enumerate}
    \item En el panel lateral inferior, haga clic en el botón \textbf{"Cerrar Sesión"}.
    \item Se mostrará un modal de confirmación.
    \item Confirme el cierre de sesión haciendo clic en \textbf{"Cerrar Sesión"} en el modal.
    \item Será redirigido a la página de inicio de sesión.
\end{enumerate}

\begin{figure}[H]
    \centering
    \includegraphics[width=0.8\textwidth]{SuperAdminImage/modal_confirmar_cerrar_sesion.png}
    \caption{Modal de confirmación para cerrar sesión}
    \label{fig:cerrar-sesion-inst}
\end{figure}

\section{Conclusión}

Este manual proporciona una guía completa para el uso del sistema SGDIS desde la perspectiva del Administrador Institucional. Como administrador institucional, tiene acceso a todas las funcionalidades del sistema dentro del alcance de su institución asignada, permitiéndole gestionar eficientemente los recursos, usuarios e inventarios bajo su responsabilidad.

\subsection{Resumen de Permisos y Restricciones}

\begin{itemize}
    \item \textbf{Puede gestionar:} Usuarios (Warehouse y User), inventarios, items, transferencias, verificaciones, préstamos y cancelaciones de su institución.
    \item \textbf{No puede gestionar:} Usuarios con rol ADMIN\_INSTITUTION, ADMIN\_REGIONAL o SUPERADMIN; regionales; instituciones.
    \item \textbf{Sin acceso a Centros:} No tiene acceso al módulo de gestión de centros/instituciones.
    \item \textbf{Alcance:} Todas sus acciones están limitadas a su institución asignada.
    \item \textbf{Transferencias:} Puede crear transferencias. Las transferencias a otras instituciones requieren aprobación.
    \item \textbf{Cancelaciones:} Puede aprobar cancelaciones de items de su institución.
\end{itemize}

\subsection{Diferencias con el Administrador Regional}

\begin{itemize}
    \item El Administrador Regional tiene acceso a todas las instituciones de su regional.
    \item El Administrador Regional puede gestionar el módulo de Centros.
    \item El Administrador Regional puede crear usuarios con rol ADMIN\_INSTITUTION.
    \item El Administrador Institucional solo tiene acceso a su propia institución.
\end{itemize}

Para obtener ayuda adicional o reportar problemas, contacte al equipo de soporte técnico del sistema, a su Administrador Regional o a un Super Administrador.

\end{document}



